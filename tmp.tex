\documentclass[UTF8]{ctexart}
\usepackage{geometry}
\usepackage{fancyhdr}
\usepackage{graphicx}
\usepackage{array}
\newcommand{\PreserveBackslash}[1]{\let\temp=\\#1\let\\=\temp}
\newcolumntype{C}[1]{>{\PreserveBackslash\centering}p{#1}}
\newcolumntype{R}[1]{>{\PreserveBackslash\raggedleft}p{#1}}
\newcolumntype{L}[1]{>{\PreserveBackslash\raggedright}p{#1}}
% \author{ARZhu}
\author{\zihao{-3}Luckyblock友情搬运}
\title{\zihao{2}数据删除信息竞赛(数据删除)}
\date{2020.9.27}
\geometry{left=3.18cm,right=3.18cm,top=2.54cm,bottom=2.54cm}
\begin{document}

\maketitle
\thispagestyle{empty}
\begin{center}
\zihao{-3}\textbf{(请选手务必仔细阅读本页内容)}
\end{center}

\textbf{一、题目概况}
\begin{center}
\begin{tabular}{*{4}{|C{9em}}|}
\hline
    中文题目名称 &东风谷早苗 &电话线 &摆渡车 \\ \hline
    英文题目与子目录名 &sanae  &telephone  &bus  \\ \hline
    可执行文件名 &sanae.exe  &telephone.exe  &bus.exe \\ \hline
    输入文件名 & sanae.in & telephone.in & bus.in \\ \hline
    输出文件名 & sanae.out & telephone.out & bus.out \\ \hline
    每个测试点时限 & 1s & 1s & 2s \\ \hline
    内存上限 & 128M & 128M & 256M \\ \hline
    测试点数目 & 10 & 13 & 20 \\ \hline
    每个测试点分值 & 10 & 不等 &5 \\ \hline
    附加样例文件 & 无 & 无 & 无 \\ \hline
    结果比较方式 & \multicolumn{3}{|c|}{全文比较(过滤行末空格及文末回车)} \\ \hline
    题目类型 & 传统 & 传统 & 传统 \\ \hline

\end{tabular}
\end{center}

\textbf{二、提交源程序程序名}
\begin{center}
\begin{tabular}{*{4}{|C{9em}}|}
\hline
    对于C++语言 & sanae.cpp & telephone.cpp & bus.cpp \\ \hline
\end{tabular}
\end{center}

%\textbf{三、优化开关}
%\begin{center}
%\begin{tabular}{*{4}{|C{9em}}|}
%\hline
%    对于C++语言 & -O2\ -lm	& -O2\ -lm & -O2\ -lm \\ \hline
%    对于C语言 & -O2\ -lm & -O2\ -lm & -O2\ -lm \\ \hline
%    对于pascal语言 & -O2 & -O2 & -O2 \\ \hline
%\end{tabular}
%\end{center}

\textbf{注意事项:}
\begin{enumerate}
    \item{文件名(程序名和输入输出文件名)必须使用英文小写。}
    \item{C++中函数main()的返回类型必须是int,程序正常结束时的返回值必须是0。}
    \item{评测时采用的机器配置为:CPU AMD Athlon II X2 260,内存1G,上述时限以此配置为准。}
\end{enumerate}

\newpage
\setcounter{page}{1}
\pagestyle{plain}
\pagenumbering{arabic}

% T1
\newpage
\section{东风谷早苗}
\begin{center}
\tt\large{(sanae.cpp)}
\end{center}
\subsection{问题描述}

在幻想乡,东风谷早苗是以高达控闻名的高中生宅巫女。   

某一天,早苗终于入手了最新款的钢达姆模型。作为最新的钢达姆,当然有了与以往不同的功能了,那就是它能够自动行走,厉害吧(好吧,我自重)。  

早苗的新模型可以按照输入的命令进行移动,命令包含’E’、’S’、’W’、’N’四种,分别对应四个不同的方向,依次为东、南、西、北。  

执行某个命令时,它会向着对应方向移动一个单位。作为新型机器人,自然不会只单单执行一个命令,它可以执行命令串。对于输入的命令串,每一秒它会按照命令行动一次。而执行完命令串最后一个命令后,会自动从头开始循环。  

向东移动,坐标改变改变为 $(X+1,Y)$;  

向南移动,坐标改变改变为 $(X,Y-1)$;  

向西移动,坐标改变改变为 $(X-1,Y)$;   

向北移动,坐标改变改变为 $(X,Y+1)$;  

在0时刻时早苗将钢达姆放置在了(0,0)的位置,并且输入了命令串。她想要知道T秒后钢达姆所在的位置坐标。

\subsection{输入}

第1行:一个字符串,表示早苗输入的命令串,保证至少有1个命令。  

第2行:一个正整数 $T$。

\subsection{输出}

第1行:两个整数,表示 $T$ 秒时,钢达姆的坐标。

\subsection{输入输出样例1}
\subsubsection{输入样例}

NSWWNSNEEWN

12

\subsubsection{输出样例}

-1 3

\subsection{约定和数据范围}

对于60\%的数据:$T \le 5\times 10^5$ 且命令串长度 $\le 5000$。

对于100\%的数据:$T \le 2\times 10^9$ 且命令串长度 $\le 5000$。

% T2
\newpage
\section{电话线}
\begin{center}
\tt\large{(telephone.cpp)}
\end{center}
\subsection{问题描述}

地震使得人间之里的电话线全部损坏,河城荷取是负责接到人间之里的负责人。  

该地区周围分布着 $n$ 个废弃的电话线杆,编号为 $1\dots n$。  
任意两根线杆之间没有电话线连接,共有 $p$ 对电话杆可以拉电话线。  
其他的由于地震使得无法连接。

第i对电线杆的两个端点分别是 $a_{i}$, $b_{i}$,它们的距离为 $l_i$。数据中每对 $(a_i,b_i)$只出现一次。  

编号为 $1$ 的电话杆已经接入了电话网络,整个人间之里的电话线全都连到了编号$n$ 的电话线杆上。也就是说,荷取的任务仅仅是找一条将 $1$ 号和 $n$ 号电线杆连起来的路径,其余的电话杆并不一定要连入电话网络。

幽香决定支援灾区免费为人间之里连接k对由荷取指定的电话线杆,对于此外的那些电话线,需要为它们付费,总费用决定于其中最长的电话线的长度(每根电话线仅连接一对电话线杆)。若需要连接的电话线杆不超过k对,那么支出为0.

请你计算一下,最少需要在电话线上花多少钱?

\subsection{输入}

输入文件的第一行包含三个数字 $n,p,k$。

第二行到第 $p+1$ 行,每行分别都为三个整数 $ai,bi,li$。

\subsection{输出}

一个整数,表示该项工程的最小支出,如果不可能完成则输出 -1。

\subsection{输入输出样例1}
\subsubsection{输入样例}

5 7 1

1 2 5

3 1 4

2 4 8

3 2 3

5 2 9

3 4 7

4 5 6

\subsubsection{输出样例}

4


\subsection{约定和数据范围}

对于 $28\%$ 的数据,$1\le n \le 10$。  

对于 $35\%$ 的数据,$1\le n\le 150$。

对于 $100\%$ 的数据,$1\le n\le 1000,1\le p\le 10^4,1\le l_i\le 10^6$。

% T3
\newpage
\section{摆渡车}
\begin{center}
\tt\large{(bus.cpp)}
\end{center}
\subsection{问题描述}

有 $n$ 名同学要乘坐摆渡车从人大附中前往人民大学,第 $i$ 位同学在第 $t_i$ 分钟去等车。只有一辆摆渡车在工作,但摆渡车容量可以视为无限大。摆渡车从人大附中出发、把车上的同学送到人民大学、再回到人大附中(去接其他同学),这样往返一趟总共花费 $m$ 分钟(同学上下车时间忽略不计)。摆渡车要将所有同学都送到人民大学。

凯凯很好奇,如果他能任意安排摆渡车出发的时间,那么这些同学的等车时间之和最小为多少呢?

注意:摆渡车回到人大附中后可以即刻出发。

\subsection{输入}

第一行包含两个正整数 $n,m$ ,以一个空格分开,分别代表等车人数和摆渡车往返一趟的时间。

第二行包含 $n$ 个正整数,相邻两数之间以一个空格分隔,第 $i$ 个非负整数   $t_i$ 代表第 $i$ 个同学到达车站的时刻。

\subsection{输出}

输出一行,一个整数,表示所有同学等车时间之和的最小值(单位:分钟)。

\subsection{输入输出样例1}
\subsubsection{输入样例}

5 1 

3 4 4 3 5 

\subsubsection{输出样例}

0

\subsubsection{样例解释1}

同学 1 和同学 4 在第 3 分钟开始等车,等待 0 分钟,在第 3 分钟乘坐摆渡车出发。摆渡车在第 4 分钟回到人大附中。

同学 2 和同学 3 在第 4 分钟开始等车,等待 0 分钟,在第 4 分钟乘坐摆渡车出发。摆渡车在第 5 分钟回到人大附中。

同学 5 在第 5 分钟开始等车,等待 0 分钟,在第 5 分钟乘坐摆渡车出发。

自此所有同学都被送到人民大学。总等待时间为 0。

\subsection{输入输出样例2}
\subsubsection{输入样例}

5 5 

11 13 1 5 5 

\subsubsection{输出样例}

4

\subsubsection{样例解释2}

同学 3 在第 1 分钟开始等车,等待 0 分钟,在第 1 分钟乘坐摆渡车出发。摆渡车在第 6 分钟回到人大附中。

同学 4 和同学 5 在第 5 分钟开始等车,等待 1 分钟,在第 6 分钟乘坐摆渡车出发。摆渡车在第 11 分钟回到人大附中。

同学 1 在第 11 分钟开始等车,等待 2 分钟;同学 2 在第 13 分钟开始等车, 等待 0 分钟。他/她们在第 13 分钟乘坐摆渡车出发。

自此所有同学都被送到人民大学。总等待时间为 4。

可以证明,没有总等待时间小于 4 的方案。

\subsection{约定和数据范围}

对于 $10\%$ 的数据,$n\le 10,m=1,0\le t_i\le 100$。

对于 $30\%$ 的数据,$n\le 20,m\le 2,0\le t_i\le 100$。

对于 $50\%$ 的数据,$n\le 500,m\le 100,0\le t_i\le 10^4$。

另有 $20\%$ 的数据,$n\le 500,m\le 10,0\le t_i\le 4\times 10^6$。

对于 $100\%$ 的数据,$n\le 500,m\le 100,0\le t_i\le 4\times 10^6$。 

\end{document}
