\documentclass[UTF8]{ctexart}
\usepackage{geometry}
\usepackage{fancyhdr}
\usepackage{graphicx}
\usepackage{array}
\usepackage{listings}
\usepackage{CJKulem}
\newcommand{\PreserveBackslash}[1]{\let\temp=\\#1\let\\=\temp}
\newcolumntype{C}[1]{>{\PreserveBackslash\centering}p{#1}}
\newcolumntype{R}[1]{>{\PreserveBackslash\raggedleft}p{#1}}
\newcolumntype{L}[1]{>{\PreserveBackslash\raggedright}p{#1}}
% \author{ARZhu}
\author{\zihao{-3}菜鸡选手友情搬运}
\title{\zihao{2}基本知识练习赛}
\date{2020.11.1}
\geometry{left=3.18cm,right=3.18cm,top=2.54cm,bottom=2.54cm}
\begin{document}
\maketitle
\thispagestyle{empty}
\begin{center}
\zihao{-3}\textbf{(请选手务必仔细阅读本页内容)}
\end{center}

\textbf{一、题目概况}
\begin{center}
\begin{tabular}{*{5}{|C{7em}}|}
\hline
    中文题目名称 &分段函数 &蚂蚁cpy &D流 &动态dp \\ \hline
    英文题目与子目录名 &fuc  &ant  &stream &not  \\ \hline
    可执行文件名 &fuc.exe  &ant.exe  &stream.exe &not.exe \\ \hline
    输入文件名 & fuc.in & ant.in & stream.in &not.in \\ \hline
    输出文件名 & fuc.out & ant.out & stream.out &not.out \\ \hline
    每个测试点时限 & 1s & 1s & 1s &2s \\ \hline
    内存上限 & 64M & 128M & 256M &512M \\ \hline
    测试点数目 & 10 & 10 & 50 & 10 \\ \hline
    每个测试点分值 & 10 & 10 &2 & 10 \\ \hline
    附加样例文件 & 无 & 无 & 有 & 无 \\ \hline
    结果比较方式 & \multicolumn{3}{|c|}{全文比较(过滤行末空格及文末回车)} \\ \hline
    题目类型 & 传统 & 传统 & 传统 & 传统 \\ \hline

\end{tabular}
\end{center}

\textbf{二、提交源程序程序名}
\begin{center}
\begin{tabular}{*{5}{|C{7em}}|}
\hline
    对于C++语言 & fuc.cpp & ant.cpp & stream.cpp &not.cpp \\ \hline
\end{tabular}
\end{center}

%\textbf{三、优化开关}
%\begin{center}
%\begin{tabular}{*{4}{|C{9em}}|}
%\hline
%    对于C++语言 & -O2\ -lm	& -O2\ -lm & -O2\ -lm \\ \hline
%    对于C语言 & -O2\ -lm & -O2\ -lm & -O2\ -lm \\ \hline
%    对于pascal语言 & -O2 & -O2 & -O2 \\ \hline
%\end{tabular}
%\end{center}

\textbf{三、优化开关}
\begin{center}
由于我比较 liang 心, O2 就不开了。
\end{center}

\textbf{注意事项:}
\begin{enumerate}
    \item{文件名(程序名和输入输出文件名)必须使用英文小写。}
    \item{C++中函数main()的返回类型必须是int,程序正常结束时的返回值必须是0。}
    \item{评测时采用的机器配置为:没有CPU 内存32kb,上述时限以此配置为准。}
\end{enumerate}

\newpage
\setcounter{page}{1}
\pagestyle{plain}
\pagenumbering{arabic}

% T1
\newpage
\section{分段函数}
\begin{center}
\tt\large{(fuc.cpp)}
\end{center}

\subsection{没啥用的背景}

据说即使证明了 $P=NP$ 或者是 $P\neq NP$, 生活也不会有什么太大的变化。

当然, 这与你能否解出本题没有什么关系。

\subsection{问题描述}

给定长度为 $n$ 的序列 $a$, 要求回答 $q$ 个询问, 第 $i$ 个询问会给定 $l_i、r_i$, 要求输出 $\max\limits_{l_i\le k \le r_i} \lbrace a_k \rbrace$, 

输出以换行隔开。

\subsection{输入}

第1行:两个数 $n$、$q$, 以空格隔开。

第2行:$n$ 个数, 第 $i$ 个数表示 $a_i$, 以空格隔开

接下来的 $q$ 行:每行 $2$ 个数 $l$、$r$, 以空格隔开, 表示一组询问。

\subsection{输出}

共 $q$ 行:每行一个数代表询问的答案, 按照输入的询问的顺序给出答案。

\subsection{输入输出样例1}

\subsubsection{输入样例}

\begin{lstlisting}
5 4
1 -100 8 10 0
1 2
5 5
2 4
1 5
\end{lstlisting}

\subsubsection{输出样例}

\begin{lstlisting}
1
0
10
10
\end{lstlisting}


\subsection{约定和数据范围}

对于20\%的数据,$1\le n \le 10^3$

对于40\%的数据,$1\le n \le 5\times 10^4$

对于80\%的数据,$1\le n \le 10^5$

对于100\%的数据,$1\le n \le 10^6$, $1\le q\le 1000$,$|a_i| \le 10^9$

% T2
\newpage
\section{蚂蚁cpy}
\begin{center}
\tt\large{(ant.cpp)}
\end{center}

\subsection{没啥用的背景}

据说 $trajan$ 发明的 $spaly$ 比 $tarjan$ 发明的 $splay$ 快很多, 插入删除操作的均摊复杂度均为 $O(n)$ 而常数仅为 $\frac{1}{1145141919810}$ 不仅如此, $spaly$ 还可以实现 $splay$ 不能实现的功能。

为什么如此强大的发明却被埋没了呢?那自然是因为社会的黑暗……不对, 其实是 $spaly$ 的实现过于繁杂, 即使是通常认为最简单的实现也需要使用量子编程语言编写超过 $2^{2^{2^{\cdots}}}$ 行代码。 还有更加困难的,那就是即便是非确定性图灵机也不能承载 $spaly$! 想要使用它,目前仅有的方法是把它放在传说中的黑暗膜法师 —— “vg·菜·tb” 的大脑里。 当然,十分爱惜身体的vgtb卿是不会随便在自己的大脑里运行奇奇怪怪的代码的, 所以目前 $spaly$ 并不可用, 被埋没也就不奇怪了。

当然,这里没有对解答本题有帮助的信息。

\subsection{问题描述}

给定 $n$ 个任务,标号为 $1\sim n$, 从时刻 $0$ 开始, 按照标号从小到大的顺序执行任务, 不能同时执行不同的任务。

要求分若干批次执行这些任务, 每一批包含的任务必须是相邻的, 在执行每一批任务之前, 都需要 $S$ 的准备时间。

任务 $i$ 需要 $T_i$ 的执行时间, 一个任务执行完成后,将稍作等待,直至所在批次任务全部执行完成, 即同一批次所有任务的执行完成时刻是相同的。

每个任务 $i$ 的费用是它的完成时刻乘以一个费用系数 $C_i$。

求在完成所有任务的情况下,可能的最小的费用和。

\subsection{输入} 

第一行包含整数 $N$。

第二行包含整数 $S$。

接下来 $N$ 行每行有一对整数,分别为 $T_i$ 和 $C_i$,以空格隔开, 表示第 $i$ 个任务完成所需的时间 $T_i$ 及其费用系数 $C_i$。

\subsection{输出}

输出一个整数,表示最小总费用。

\subsection{输入输出样例1}

\subsubsection{输入样例}

\begin{lstlisting}
5
1
1 3
3 2
4 3
2 3
1 4
\end{lstlisting}

\subsubsection{输出样例}

\begin{lstlisting}
153
\end{lstlisting}


\subsection{约定和数据范围}

对于20\%的数据, $1\le n\le 20$

对于50\%的数据, $1\le n\le 300$

另外有20\%的数据, $S=0$

对于100\%的数据, $1\le n\le 5000$, $0\le S\le 50$, $1\le T_i,C_i\le 100$

% T3
\newpage
\section{D流}
\begin{center}
\tt\large{(stream.cpp)}
\end{center}

\subsection{没啥用的背景}

WARNING:数据删除残留

\sout{题目不算有趣,不妨来点苏联政治笑话}。

WARNING:数据删除残留

\sout{从前有个}

(数据删除)

当然,这个笑话不是本题的重点。

\subsection{问题描述}

大学生 JOI 君每天乘坐巴士走读。

JOI 君的家和学校都在 IOI 市内。IOI 市内共有 $N$ 个巴士站点,编号为 $1\sim N$ ,离 JOI 家最近的站点为 $1$ 号站点,离大学最近的站点为 $N$ 号站点。

IOI 市内共有 $M$ 辆巴士,每辆巴士一天只跑一次,从某一时刻某一停靠点出发,在某一时刻到达另一个站点。运行途中不可以下车。

JOI 君每天要乘坐一次以上的巴士到达学校。我们可以无视 JOI 君换车的时间,换言之,为了换乘某个时刻从某个停靠点出发的巴士,只需要在该巴士的出发时间或之前到达站点就可以了。此外,多次在某个站点换乘也是可以的。

在这样的条件下,JOI 君想知道自己应该何时从家中出发才能按时赶到学校。然而,学校每天开始上课的时间都不同。在某 $Q$ 天里,每天到达 $N$ 号站点的最晚时间都是已知的,JOI 君想知道,他最晚何时到达 $1$ 号站点才能及时到校。

现在给你巴士的运营信息,以及这 $Q$ 天里每天到达 $N$ 号站点的最晚时间,请你求出每天 JOI 君最晚何时到达 $1$ 号站点。

\subsection{输入}

第一行两个空格分隔的正整数 $N$ 和 $M$ ,表示 IOI 市内有 $N$ 个巴士站点和 $M$ 辆巴士。

接下来 $M$ 行,第 $i$ 行 $(1\le i\le M)$ 有四个空格分隔的整数 $A_i$, $B_i$, $X_i$, $Y_i$ $(1\le A_i,B_i\le N, A_i\neq B_i)$,表示第 $i$ 辆巴士在时刻 $X_i$ 从停靠点 $A_i$ 出发,在时刻 $Y_i$ 到达停靠点 $B_i$ 。时刻从半夜 12 点开始计算,单位为毫秒。

接下来一行一个整数 $Q$ ,含义如题目中所示。

接下来 $Q$ 行,第 $i$ 行 $(1\le i\le Q)$ 有一个整数 $L_i$ ,表示第 $i$ 天最迟 $L_i$ 时刻到达 $N$ 号站点。

\subsection{输出}

输出 $Q$ 行,第 $i$ 行 $(1\le i\le Q)$ 一个整数,表示 JOI 君第 $i$ 天最迟到达 $1$ 号站点的时刻。 如果无法在时限内到达,输出 $-1$ 。

\subsection{输入输出样例1}


\subsubsection{输入样例}

\begin{lstlisting}
5 6
1 2 10 25
1 2 12 30
2 5 26 50
1 5 5 20
1 4 30 40
4 5 50 70
4
10
30
60
100
\end{lstlisting}


\subsubsection{输出样例}

\begin{lstlisting}
-1
5
10
30
\end{lstlisting}


\subsubsection{样例解释1}

无法在时刻 10 之前到达 5 号车站。 为了在时刻 30 到达,您可以在时刻 5 乘坐 4 号车。 为了在时刻 60 到达,您可以执行以下操作。

- 在时刻 10 乘坐 1 号车。

- 在时刻 25 到达 2 号车站,等待 1 ms,转 3 号车。

- 在时刻 50 抵达 5 号车站。

为了在时间 100 到达,您可以执行以下操作。

- 在时刻 30 乘坐 5 号车。

- 在时刻 40 到达 4 号车,等待 10 ms,转 6 号车。

- 在时刻 70 抵达 5 号车站。

\subsection{输入输出样例2}
\subsubsection{输入样例}

\begin{lstlisting}
3 8
1 2 1 5
1 3 0 1
1 3 2 8
2 3 2 3
2 3 3 4
2 3 4 5
2 3 5 6
2 3 6 7
6
3
4
5
6
7
8
\end{lstlisting}


\subsubsection{输出样例}

\begin{lstlisting}
0
0
0
1
1
2
\end{lstlisting}

\subsection{约定和数据范围}

对于15\%的数据, $N,M\le 2000$

对于另外35\%的数据, $Q=1$

对于所有数据, $2\le N\le 10^5,1\le M\le 3\times 10^5,0\le X_i<Y_i<86400000(=24\times 60\times 60\times 1000),1\le Q\le 10^5,0\le L_i<86400000$。



% T4
\newpage
\section{动态DP}
\begin{center}
\tt\large{(not.cpp)}
\end{center}

\subsection{没啥用的背景}

来两碗不算很差的鸡汤。

“我其实很怀疑「智商」这概念的提出者自己的智商,他有什么资格来评价其他人的智商?人的能力是多种多样的。用仅仅一个数字就可以评价别人头脑的能力,似乎只是满足了某些人不切实际的妄想。
”

“我还记得很久以前有人跟我说的话,自己选择的路,跪着也要走完。朋友们,虽然这个世界日益浮躁起来,只要能够为了当时纯粹的梦想和感动坚持努力下去,不管其它人怎么样,我们也能够保持自己的本色走下去。”

感觉到力量了么?那么请用乱搞……不,智慧,愉快地切掉这道题吧。

\subsection{问题描述}

给定一个 $n$ 个点的树, 每个点 $i$ 都有一个正整数点权 $A_i$, 你需要支持以下两种操作:

1、询问点 $x$ 和点 $y$ 之间的路径上的所有点(包括点 $x$ 和点 $y$ )的点权是否构成一个从 $1$ 开始的排列(即若这条链长度为 $len$ ,那么问点权集合是否为 $\lbrace 1,2,\cdots,len\rbrace$ )。

2、将 $A_x$ 修改为 $y$

\subsection{输入}

第一行一个正整数 $T$ 表示数据组数。

接下来一行输入两个正整数 $n,q$ 表示树的点数和询问个数。

接下来一行 $n$ 个正整数,第 $i$ 个正整数表示 $A_i$ 的初值。

接下来 $n−1$ 行每行两个正整数 $u,v$ 表示树上的一条边 $(u,v)$ 。

接下来 $q$ 行每行三个正整数 $tp,x,y$ 表示一个操作,其中 $tp$ 表示操作种类。

\subsection{输出}

对于每一个操作 $1$ 如果符合条件,输出 Yes ,否则输出 No 。

\subsection{输入输出样例1}


\subsubsection{输入样例}

\begin{lstlisting}
1
10 5
1 2 3 4 5 6 7 8 9 10
1 2
1 3
2 4
2 5
3 9
4 10
5 6
5 7
5 8
1 4 3
1 10 3
2 10 5
1 10 3
1 5 3
\end{lstlisting}

\subsubsection{输出样例}

\begin{lstlisting}
Yes
No
Yes
No
\end{lstlisting}

\subsection{约定和数据范围}

对于10\%的数据, $1\le n,q\le 1000$

对于另外40\%的数据, 对于输入的每一条边, 保证 $u=v+1$

对于100\%的数据,保证 $1\le T\le 10$, $1\le n,q\le 2.5\times 10^5$, $1\le \sum n,\sum q\le 5\times 10^5$,$1\le u,v,x,y,A_i \le n$,$1\le tp\le 2$

\end{document}
